\documentclass[12pt]{article}
\pdfoutput=1
\usepackage{bm}% bold math

%\usepackage{anysize}
\usepackage[colorlinks,hyperindex, urlcolor=blue, linkcolor=blue,citecolor=black, linkbordercolor={.7 .8 .8}]{hyperref}
\usepackage{graphicx}
%\usepackage{tabularx}
\usepackage{amsfonts}
\usepackage{amsmath}
\usepackage{amssymb}
\usepackage{amsbsy}
\usepackage{tikz}
\usepackage[margin=0.5in]{geometry}
\usepackage{nicefrac}
\usepackage{subcaption}
\usetikzlibrary{arrows,shapes,positioning}
\newenvironment{psmallmatrix}
  {\left[\begin{matrix}}
  {\end{matrix}\right]}
  
 \usepackage{listings}
\usepackage{color}


\begin{document}
\noindent This document is meant to help me organize my thoughts. I kept losing track of indices and factors while doing this by hand. I want expressions which can be easily and efficiently implemented.\\

\noindent Index convention:
\begin{equation*}
\begin{split}
a,b,c,d,... &= \text{particles},\\
i,j,k,l,... &= \text{holes},\\
p,q,r,s,... &= \text{either}.\\
\end{split}
\end{equation*}


\noindent One- and two- body operators have the following form:
\begin{equation}
A^{(1)} = \sum_{pq} A_{pq} :a_p^\dagger a_q:,
\end{equation}
\begin{equation}
A^{(2)} = \frac{1}{4} \sum_{pqrs} A_{pqrs} :a_p^\dagger a_q^\dagger a_s a_r:,
\end{equation}
\noindent where the colons indicate normal ordering. I want to write a function that returns the zero-, one-, and two-body components of the commutator
\begin{equation}
[A^{(1)}+A^{(2)},B^{(1)}+B^{(2)}].
\end{equation}
I will name these components $C^{(0)}$, $C^{(1)}$, and $C^{(2)}$, respectively. Note: the three-body component will be omitted. Since this commutator is equivalent to the following sum of commutators
\begin{equation}
[A^{(1)},B^{(1)}]+[A^{(1)},B^{(2)}]+[A^{(2)},B^{(1)}]+[A^{(2)},B^{(2)}],
\end{equation}
we can use the fundamental commutator expressions to evaluate each term. For reference, I will write the relevant expressions here:

% 1B-1B
\begin{align}
[A^{(1)},B^{(1)}]^{(0)} &= \sum_{pq}(n_p-n_q)A_{pq}B_{qp},\\
[A^{(1)},B^{(1)}]^{(1)} &= \sum_{pq}\sum_{r} :a_p^\dagger a_q:(A_{pr}B_{rq}-B_{pr}A_{rq}),
\end{align}

% 1B-2B
\begin{align}
[A^{(1)},B^{(2)}]^{(1)} &= \sum_{pq}\sum_{rs} :a_p^\dagger a_q: (n_r-n_s) A_{rs}B_{sprq},\\
[A^{(1)},B^{(2)}]^{(2)} &= \frac{1}{4} \sum_{pqrs}\sum_t :a_p^\dagger a_q^\dagger a_s a_r : \{ (1-P_{pq})A_{pt}B_{tqrs} -(1-P_{rs})A_{tr}B_{pqts}\},\\
[A^{(2)},B^{(1)}]^{(1)} &= -[B^{(1)},A^{(2)}]^{(1)} = \sum_{pq}\sum_{rs} :a_p^\dagger a_q: (n_s-n_r) B_{rs}A_{sprq},\\
[A^{(2)},B^{(1)}]^{(2)} &= -[B^{(1)},A^{(2)}]^{(2)} = \frac{1}{4} \sum_{pqrs}\sum_t :a_p^\dagger a_q^\dagger a_s a_r : \{(1-P_{rs})B_{tr}A_{pqts} -  (1-P_{pq})B_{pt}A_{tqrs}\},
\end{align}

% 2B-2B
\begin{align}
[A^{(2)},B^{(2)}]^{(0)} &= \frac{1}{4} \sum_{pqrs} n_pn_q\bar{n}_r \bar{n}_s (A_{pqrs}B_{rspq}-B_{pqrs}A_{rspq}),\\
[A^{(2)},B^{(2)}]^{(1)} &= \frac{1}{2} \sum_{pq}\sum_{rst} :a_p^\dagger a_q: (\bar{n}_r \bar{n}_s n_t + n_r n_s \bar{n}_t) (A_{tprs}B_{rstq}-B_{tprs}A_{rstq}),\\
[A^{(2)},B^{(2)}]^{(2)} &= \frac{1}{4} \sum_{pqrs}\sum_{tu} :a_p^\dagger a_q^\dagger a_s a_r : \{ \frac{1}{2} (A_{pqtu}B_{turs}-B_{pqtu}A_{turs})(1-n_t-n_u)\\ &\ \ \ +  (n_t-n_u)(1-P_{ij})(1-P_{kl})A_{tpur}B_{uqts} \}
\end{align}

Now, we can simplify this a bit because many of the terms in the summations vanish. Using the index notation mentioned above, we have

\begin{align}
C^{(0)} &= \sum_{pq}(n_p-n_q)A_{pq}B_{qp} + \frac{1}{4} \sum_{pqrs} n_pn_q\bar{n}_r \bar{n}_s (A_{pqrs}B_{rspq}-B_{pqrs}A_{rspq}),
\end{align}






\end{document}
